\documentclass{article}
\usepackage[utf8]{inputenc}
\usepackage{hyperref}

\begin{document}
\subsection*{Question 1}
    \begin{itemize}
        \item 
        This is a well known problem known as \href{https://en.wikipedia.org/wiki/Coupon_collector's_problem}{The Coupon collector's Problem}.
            \begin{itemize}
            \item
            Estimating the expected number of dice rolls using R
                    \begin{verbatim}
## function that simulates rolling a n-sided die
diceroll <- function(n) {
  die <- 1:n
  result <- sample(die, 1)
  result
}
## function that simulates rolling a 12-sided die until all possible outcomes 
## have happened at least once
rollcollector <- function() { 
  rolls <- diceroll(12)
  result <- 1
  while(length(unique(rolls))<12) {
    rolls <- append(rolls, diceroll(12))
    result <- result + 1
  }
  result
}
                    \end{verbatim}
                    \begin{itemize}
                        \item 
                            Expected value of 100 trials:
                            \begin{verbatim}
> trials <- 100
> simlist <- replicate(trials, rollcollector())
> mean(simlist)
[1] 40.22
                            \end{verbatim}
                        \item 
                            Expected value of 1000000 trials:
                            \begin{verbatim}
> trials <- 1000000
> simlist <- replicate(trials, rollcollector())
> mean(simlist)
[1] 37.24014
                            \end{verbatim}
                    \end{itemize}
                \item
                In order to solve this problem analytically, we will define the following random variables:
        \(X_i\): number of dice rolls needed until  the \(i^{th}\) outcome has occurred\\
        \[X_i = X_1 + X_2 + X_3 + \cdots \]
        \(Y_i\): number of dice rolls between the \((i-1)^{th}\) occurring for the first time and the \(i^{th}\) outcome occurring for the first time\\
        \[Y_i = X_{i+1}-X_i\]
        Our goal is to compute \(E(X_{12})\), the expected number of dice rolls until all possible outcomes of our 12-sided die have occurred at least once. Using the definitions above, we can see that
        \[X_{12} = \sum_{i=0}^{11} Y_i\]
        This will prove to be useful soon enough.\\
        Now, let's find the P.M.F. of \(Y_i\), starting with finding the probability of \(Y_1\). There is a total of 12 possible outcomes, and we have already seen 1. So there are \(12-1\) potential sides left. Therefore
        \[Y_1 = Geo(\frac{12-1}{12})\]
        Generalising this, we get
        \[Y_i = Geo(\frac{12-i}{12})\]
        Since \(Y_i\) follows a geometric distribution, we know that \(E(Y_i)\) is simply the inverse of the P.M.F.
        \[E(Y_i) = \frac{12}{12-i}\]
        Now we can compute \(E(X_{12})\)
        \[E(X_{12}) = E[\sum_{i=0}^{11} Y_i]\]
        \[= \sum_{i=0}^{11}E( Y_i)\]
        \[= \sum_{i=0}^{11}\frac{12}{12-i}\]
        \[= 12\sum_{i=0}^{11}\frac{1}{12-i}\]
        \[= \frac{86021}{2310}\]
        \[\approx 37.2385\]
            \end{itemize}
        \item
        Having already completed the first part of the problem, this part is quite simple as we only need to expand our code to include two dice
            \begin{verbatim}
## expanding rollcollector 
rollcollector2 <- function() {
  rolls <- sum(replicate(2, diceroll(6)))
  result <- 1
  while(length(unique(rolls))<11) {
    rolls <- append(rolls, sum(replicate(2, diceroll(6))))
    result <- result + 1
  }
  result
}
trials <- 10000
simlist2 <- replicate(trials, rollcollector2())
            \end{verbatim}
            \begin{itemize}
                \item Expected value of 100 trials
                \begin{verbatim}
> trials <- 100
> simlist2 <- replicate(trials, rollcollector2())
> mean(simlist2)
[1] 62.77       
                \end{verbatim}
                \item Expected value of 100000 trials
                \begin{verbatim}
> trials <- 100000
> simlist2 <- replicate(trials, rollcollector2())
> mean(simlist2)
[1] 61.27343
                \end{verbatim}
            \end{itemize}
    \end{itemize}
         

\end{document}
